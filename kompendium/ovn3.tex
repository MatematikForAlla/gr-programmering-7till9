\chapter{Lektion 3}

% Föreläsning 4
\section{Datatypen list}

List är ytterligare en datatyp i Python, den kan användas för att lagra ett
godtyckligt antal värden eller objekt i. T.ex. när man samlar data och har en
datamängd som är större än ett (1) element.

En lista har ett antal metoder, en slags funktion definierad på objektet
självt, nedan följer en sammanfattning av några av dessa. Om vi skapar en tom
lista \code{l}, genom att skriva \code{l = list()} alternativt \code{l = []},
kan vi använda följande metoder:

\begin{description}
\item[\code{l.append(e)}] Lägg till \code{e} som ett element sist i listan.
\item[\code{l.extend(l2)}] Lägg till listan \code{l2} sist i listan.
\item[\code{l.insert(p, e)}] Sätt in \code{e} som ett element framför elementet
	på positionen \code{p}.
\item[\code{l.index(e)}] Returnera det lägsta möjliga index för elementet som
	matchar \code{e}.
\item[\code{l.remove(e)}] Ta bort första förekomsten av ett element som matchar
	\code{e}.
\end{description}

Man kan också skapa en lista som redan innehåller några element genom koden
\code{l = [1, 2, 3, "hej"]}.

Några exempel i Python-tolken:
\begin{lstlisting}[style=text]
Python 2.2.3 (#1, Jan  5 2005, 16:36:30)
[GCC 3.4.2] on sunos5
Type "help", "copyright", "credits" or "license" for more information.
>>> l = [1, 2, 3]
>>> print l
[1, 2, 3]
>>> print l[0]
1
>>> print l[2]
3
>>> l.append(5)
>>> print l
[1, 2, 3, 5]
>>> l.extend([2,3])
>>> print l
[1, 2, 3, 5, 2, 3]
>>> l.insert(0, 10)
>>> print l
[10, 1, 2, 3, 5, 2, 3]
>>> l.index(3)
3
>>> l.remove(3)
>>> l.index(3)
5
>>>
\end{lstlisting}



\section{Datatypen dictionary}

Ännu en datatyp hos Python, denna kan användas till att para ihop saker. T.ex.
namn och telefonnummer, man kan söka på ett namn och få ut ett telefonnummer.

Om vi skapar ett dictionary \code{d}, genom koden \code{d = \{\}}, kan vi sedan
använda följande metoder:

\begin{description}
\item[\code{d.keys()}] Returnerar en lista med alla lagrade nycklar.
\item[\code{d.values()}] Returnerar en lista med alla lagrade värden.
\item[\code{k in d}] Evalueras till sant (\code{True}) om nyckeln \code{k}
	finns i \code{d}.
\end{description}

För att skapa en dictionary innehållandes några par skriver man följande:
\code{d = \{"nyckel" : "värde", "programmering" : "python", 1 : -1\}}.

Några exempel:
\begin{lstlisting}[style=text]
>>> d = {"hej" : "svejs", "kth" : "kungl tekn högskol"}
>>> d.keys()
['kth', 'hej']
>>> d.values()
['kungl tekn h\xf6gskol', 'svejs']
>>> "kth" in d
1
>>> "su" in d
0
>>>
\end{lstlisting}



\section{Lite mer om strängar}

Strängar har, likt listorna och dictionaries, också metoder. Några av dessa
metoder beskrivs här:

\begin{description}
\item[\code{s.split(c)}] Delar upp strängen vid varje \code{c}, returnerar en
	lista med alla delarna.
\item[\code{s.strip(c)}] Returnerar en sträng där alla \code{c} tagits bort från
	början och slut av strängen.
\item[\code{s.rstrip(c)}] Samma som strip(), men behandlar enbart slutet av
	strängen -- början lämnas orörd.
\end{description}

Några exempel på deras användning:
\begin{lstlisting}[style=text]
>>> "hej svejs i lingonskogen".split()
['hej', 'svejs', 'i', 'lingonskogen']
>>> s = "hej svejs i lingonskogen"
>>> print s
hej svejs i lingonskogen
>>> s.split()
['hej', 'svejs', 'i', 'lingonskogen']
>>> s.split('e')
['h', 'j sv', 'js i lingonskog', 'n']
>>> s = "   hej svejs i lingonskogen    "
>>> print s.strip()
hej svejs i lingonskogen
>>> print s.rstrip()
   hej svejs i lingonskogen
>>> s = "...hej svejs i lingonskogen...."
>>> print s.strip('.')
hej svejs i lingonskogen
>>> print s.rstrip('.')
...hej svejs i lingonskogen
>>>
\end{lstlisting}



\section{Felhantering}

Förutom den grundläggande felhantering man kan skriva själv i ett program finns
det vissa fel som man inte lika enkelt kan skydda sig emot, där erbjuder Python
något som kallas för exceptions. Ett \emph{exception} är en slags signal som
''sänds ut'' när ett fel uppstår, detta exception kan man sedan fånga upp på
ett speciellt ställe i koden och där behandla. Man hanterar exceptions enligt
följande:
\begin{lstlisting}
try:
   # kod som kan ge upphov till exceptions
except <typ0>:
   # kod som ska köras när ett exception av typ <typ0> fångas
except <typ1>:
   # körs när man fångar ett exception av typ <typ1>
except:
   # kod som körs för alla andra exceptions
\end{lstlisting}
Det är helt frivilligt hur många \code{except} man använder sig av, men man
måste ha minst ett (1).

Några exempel:
\begin{lstlisting}[style=text]
Python 2.6.2 (r262:71600, Aug 12 2009, 11:11:06)
[GCC 3.3.5 (propolice)] on openbsd4
Type "help", "copyright", "credits" or "license" for more information.
>>> try:
...     int("hej")
... except:
...     print "exception"
...
exception
>>> int("hej")
Traceback (most recent call last):
  File "<stdin>", line 1, in <module>
ValueError: invalid literal for int() with base 10: 'hej'
>>> try:
...     int("hej")
... except ValueError:
...     print "valueerror"
... except:
...     print "exception"
...
valueerror
>>> def f(a,b):
...     return float(a)/float(b)
...
>>> f("5","3")
1.6666666666666667
>>> f("hej", "du")
Traceback (most recent call last):
  File "<stdin>", line 1, in <module>
  File "<stdin>", line 2, in f
ValueError: invalid literal for float(): hej
>>> f("5","0")
Traceback (most recent call last):
  File "<stdin>", line 1, in <module>
  File "<stdin>", line 2, in f
ZeroDivisionError: float division
>>> def f(a,b):
...     try:
...             return float(a)/float(b)
...     except ValueError:
...             print "valueerror"
...     except ZeroDivisionError:
...             print "zerodivisionerror"
...
>>> f("5","4")
1.25
>>> f("a","b")
valueerror
>>> f("5","0")
zerodivisionerror
>>>
\end{lstlisting}



\section{Filhantering}

Ibland kan man vilja ''komma ihåg'' saker mellan körningarna av programmet, när
alla variabler försvinner när programmet avslutas, eller läsa in stora mängder
data som är otympligt att mata in från tangentbordet. Där kommer filhantering
lämpligt in i bilden. Vi kan skriva det vi vill komma ihåg till en fil, och
sedan läsa in den nästa gång vi startar programmet.

För att öppna en fil i Python använder man sig av funktionen \code{open()},
denna funktion tar som första argument en sträng, som är
sökvägen\footnote{Sökvägen kan vara absolut eller relativ. Om den är relativ
utgår man från den mapp som programfilen finns i.} till filen som ska öppnas. 
Det andra argumentet är sättet vi vill öppna filen på, för läsning eller
för skrivning.

Open returnerar ett objekt som representerar filen. Objektet har några metoder
för att kunna läsa och skriva data från respektive till filen. Om vi öppnar
filen \emph{testfil.txt} för läsning genom
\begin{lstlisting}
fil = open("testfil.txt", "r")
\end{lstlisting}
kan vi använda oss av följande metoder för att utföra olika handlingar på
filen:
\begin{description}
	\item[\code{text = fil.read()}] Läs in hela filens innehåll till variabeln
		\code{text}.
	\item[\code{rader = fil.readlines()}] Läs in hela filen, spara alla rader
		som separata element i listan \code{rader}.
	\item[\code{rad = fil.readline()}] Läs in nästa rad till variabeln
		\code{rad}.
\end{description}
När vi inte längre behöver använda filen stänger vi den med \code{fil.close()}.

Om vi istället öppnar filen för skrivning genom
\begin{lstlisting}
fil = open("testfil.txt", "w")
\end{lstlisting}
kan vi använda oss av följande metoder:
\begin{description}
	\item[\code{fil.write(\"hello file\n\")}] Skriv \emph{hello file} till
		filen, följt av en radbrytning.
	\item[\code{fil.writelines(rader)}] Skriv alla strängar i listan rader till
		filen, notera att de inte skrivs på varsin rad om de inte själva
		innehåller radbrytningstecken.
\end{description}
När vi inte längre behöver använda filen stänger vi den med \code{fil.close()}.

Några exempel från terminalen och Python-tolken:
\begin{lstlisting}[style=text]
dbosk@my:/tmp\$ ls testfil.txt
testfil.txt: No such file or directory
dbosk@my:/tmp\$ python
Python 2.2.3 (#1, Jan  5 2005, 16:36:30)
[GCC 3.4.2] on sunos5
Type "help", "copyright", "credits" or "license" for more information.
>>> fil = open("testfil.txt", "w")
>>> fil.write("hello file\nen rad till")
>>> fil.close()
>>> \^D
dbosk@my:/tmp\$ ls testfil.txt
testfil.txt
dbosk@my:/tmp\$ python
Python 2.2.3 (#1, Jan  5 2005, 16:36:30)
[GCC 3.4.2] on sunos5
Type "help", "copyright", "credits" or "license" for more information.
>>> fil = open("testfil.txt", "r")
>>> print fil.read()
hello file
en rad till
>>> fil.close()
>>> fil = open("testfil.txt", "r")
>>> print fil.readline()
hello file

>>> print fil.readline()
en rad till
>>> print fil.readline()

>>> \^D
dbosk@my:/tmp\$ 
\end{lstlisting}



% Föreläsning 5
\section{Klass och objekt}

...
% som vilken datatyp som helst (obj i lista)
% str, dictionary och list är klasser


\subsection{Instantiering}

...


\subsection{Instansvariabler och instansmetoder}

...


\subsection{Konstruktor}

...


\subsection{self}

...


\subsection{Inkapsling}

...



\section{Programexempel}

Två programexempel följer. Det första, \code{O3.py}, behandlar felhantering,
listor och dictionaries. När man skriver in kommandon i terminalen, t.ex.
\code{ls} eller \code{python}, kollar terminalprogrammet (kallat en shell) upp
kommandot i en dictionary-liknande struktur för att veta hela sökvägen till 
programmet som skall köras. Detta enkla lilla program gör samma sak. Det 
låter användaren skriva in ett kommando, sedan kollar det upp sökvägen i ett 
dictionary och skriver ut den.
\lstinputlisting[language=Python]{O3.py}

Det andra programmet, \code{O3-sten-version3.py}, behandlar klasser samt in-
och utmatning till fil. Det är
ett bankprogram som håller ordning på en massa bankkonton, tillåter att man
sätter in pengar och kollar saldot.
\lstinputlisting[language=Python]{O3-sten-version3.py}
