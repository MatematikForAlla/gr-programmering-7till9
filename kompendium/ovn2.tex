\chapter{Lektion 2}

\section{Datatypen bool}

\code{bool} är en datatyp som, till skillnad från t.ex. \code{int}, kan ta
enbart två olika värden -- \code{True} och \code{False} (skrivs också
vanligtvis som 1 respektive 0)\footnote{Detta genom typkonvertering.}.



\section{Logiska operatorer}

På denna datatyp \verb'bool' finns några operatorer definierade, likt
\verb'+ - * /' för flyttal. Dessa operatorer är \verb'and or not'. \verb'not'
är en operator som skiljer sig något från de tidigare operatorerna vi tittat
på, den är unär, d.v.s. den opererar enbart på en operand, till skillnad från
binära operatorer som opererar på två operander. Vi sammanfattar
funktionaliteten hos dessa operatorer i tabellerna \ref{tbl:not} och
\ref{tbl:andor}.

\begin{table}[h]
	\begin{tabular}{ll}
	Operation & Resultat \\
	\hline
	not True & False \\
	not False & True
	\end{tabular}
\label{tbl:not}
\caption{Sanningtabell för Not}
\end{table}

\begin{table}[h]
	\begin{tabular}{lcll}
	and & \vline & True & False \\
	\hline
	True & \vline & True & False \\
	False & \vline & False & False
	\end{tabular}
	\begin{tabular}{lcll}
	or & \vline & True & False \\
	\hline
	True & \vline & True & True \\
	False & \vline & True & False
	\end{tabular}
\label{tbl:andor}
\caption{Sanningstabeller för And och Or}
\end{table}

Några exempel från Python-tolken:
\begin{lstlisting}[style=text]
Python 2.2.3 (#1, Jan  5 2005, 16:36:30)
[GCC 3.4.2] on sunos5
Type "help", "copyright", "credits" or "license" for more information.
>>> True or False
1
>>> True and False
0
>>> not True
0
>>> not False
1
>>> not True and False
0
>>> not (True and False)
1
>>> not False and True
1
>>> not (False and True)
1
>>>
\end{lstlisting}



\section{Jämförelseoperatorer}

Precis som de aritmetiska och logiska operatorerna finns det operatorer för
jämförelse definierade för de vanliga datatyperna i Python. Dessa är
\verb'== < > <= >= !='. De används på samma sätt som de aritmetiska
operatorerna, men skiljer sig genom att de ger upphov till ett värde som är av
boolesk typ (\verb'bool') istället för ett värde av samma typ som operanderna.
Några exempel:
\begin{lstlisting}[style=text]
>>> 5+5
10
>>> 5 < 5
0
>>> 3< 5
1
>>> 5==5
1
>>> 4 <= 5
1
>>> 5 <= 5
1
>>>
\end{lstlisting}



\section{Villkorssatser}

Villkorssatser används för att kontrollera programflödet, d.v.s. i vissa fall
vill vi att programmet ska göra en sak och i andra fall något annat. För detta
behöver vi olika villkor.

Ett villkor är ett uttryck som kan evalueras till antingen \code{True} eller
\code{False}, exempelvis \code{x < 5} eller \code{x < 5 and x != 0}.

En typ av villkorssats är \verb'if-elif-else'-satsen. Den fungerar på följande
sätt:
\begin{lstlisting}
if <villkor>:
    # kod
# en eller flera elif-satser
elif <annat_villkor>:
    # kod
else:
    # kod
\end{lstlisting}
Notera att man inte nödvändigtvis behöver ha med varken \code{elif} eller
\code{else}, och man kan ta med hur många \code{elif} man önskar.

När något villkor är sant ignoreras alla efterföljande \code{elif} och
\code{else} även om deras villkor skulle vara sanna. Ett exempel:
\begin{lstlisting}[style=text]
>>> x=5
>>> if x < 5:
...     print "less than 5"
... elif x < 10:
...     print "less than 10"
... else:
...     print "too large"
...
less than 10
>>>
\end{lstlisting}



\section{Slingor}

Slingor används för att få dynamiska upprepningar i programmet, t.ex. när vi
vill läsa in en fil eller gå igenom en lista. Det finns två olika
loop-konstruktioner som tas upp här, de är for- och while-satserna.

for-satsen fungerar på följande sätt:
\begin{lstlisting}
for <variabel> in <lista>:
   # kod
\end{lstlisting}
Koden i for-loopen kommer att köras en gång för varje element i listan, varje
gång kommer variabeln ha värdet av ett element i listan. Ett exempel i
python-tolken:
\begin{lstlisting}[style=text]
>>> for x in [1, 2, 3]:
...     print "hej"*x
...
hej
hejhej
hejhejhej
>>> for i in range(10):
...     print i,
...
0 1 2 3 4 5 6 7 8 9
>>>
\end{lstlisting}

Den andra loop-konstruktionen är while-satsen. Den kör sin kod så länge ett
givet villkor är sant. Den börjar med att kolla villkoret, om det är sant körs
koden, annars fortsätter exekveringen direkt efter while-loopen.
\begin{lstlisting}
while <villkor>:
   # kod
\end{lstlisting}
Notera att den enbart kollar villkoret en gång per varv, så hela koden körs
trots att villkorets värde ändras under körningens gång. Exempel:
\begin{lstlisting}[style=text]
> i=0
>>> while i<10:
...     print i,
...     i += 1
...
0 1 2 3 4 5 6 7 8 9
>>>
\end{lstlisting}



\section{Programexempel}

Det första programmet, \verb'O2-sten.py', är ett litet program som tar upp de
grundläggande begreppen för denna övning, och däribland också nästlade slingor.
\lstinputlisting[language=Python]{O2-sten.py}

Det andra programmet, \verb'O2.py', är ett större program som beräknar portot
för angivna brev- och paketvikter.
\lstinputlisting[language=Python]{O2.py}
