\chapter{Lektion 1}

\section{Datatyper}

Man kan säga att nästan allt i Python är av någon datatyp. Dessa datatyper
håller Python själv reda på, men man måste även som programmerare hålla koll
på dessa för att programmet ska fungera som man vill.

De datatyper som Python definierar är bl.a.
\begin{description}
	\item[int] Heltal (Integers),
	\item[float] Flyttal (Floating point numbers),
	\item[str] Strängar (Strings).
\end{description}



\section{Aritmetiska operatorer}

På dessa datatyper finns operatorer definierade, t.ex. \verb'+' är en operator
definierad för heltal. Andra operatorer som finns är \verb'+ - * / // %'. Några
exempel (som körs i Pythons tolk från terminalen):
\begin{lstlisting}[style=text]
\$ python
Python 2.2.3 (#1, Jan  5 2005, 16:36:30)
[GCC 3.4.2] on sunos5
Type "help", "copyright", "credits" or "license" for more information.
>>> 5+5
10
>>> 3*2
6
>>> 5//3
1
>>> 5\%3
2
>>> "hej"+"svejs"
'hejsvejs'
>>> "hej"*3
'hejhejhej'
>>>
\end{lstlisting}

Alla operatorer är dock inte definierade för alla datatyper, en mycket kort
och ej fullständig sammanfattning är:
\begin{description}
	\item[Heltal] kan använda operatorerna \verb'+ - * / // %',
	\item[Flyttal] kan använda operatorerna \verb'+ - * /', heltalsdivision
	och modulo går inte att beräkna för flyttal,
	\item[Strängar] har bara operatorerna \verb'+ *'.
\end{description}



\section{Konstanter}

För att förenkla beräkningar etc. har vi något som kallas för konstanter.
Dessa används för att underlätta när vi har återkommande värden, t.ex. \(\pi\).
En konstant i Python skapas på följande vis \code{PI=3.14}, och kan sedan
användas genom att man skriver \code{PI} istället för \code{3.14} överallt i
sin kod. En exempelkörning i Python kan se ut enligt följande.
\begin{lstlisting}[style=text]
>>> PI=3.14
>>> 2*PI
6.2800000000000002
>>>
\end{lstlisting}

Konstanter kan också användas för vissa restriktioner som man kanske har i sin
kod, t.ex. att ett namn måste vara 32 tecken. Om man senare kommer på att
namnet borde kunna vara 64 tecken långt behöver man bara ändra värdet på
konstanten\footnote{Med att ändra värde på konstanten menare jag att man ändrar
i koden, värdet på en konstant kan inte ändras under programkörning.} istället
för att leta upp alla ställen i koden där det används.



\section{Identifierare}

Vilka regler gäller då för namnet på konstanterna? De får bestå av bokstäver,
siffror och även understreck (\verb'_'), men de får dock inte börja med
siffror. De får heller inte vara något av följande reserverade ord:
\begin{verbatim}
    and assert break class continue def del elif
    else except exec finally for from global if
    import in is lambda not or pass print raise
    return try while
\end{verbatim}
Viktigt att tänka på är att man skiljer på gemener och versaler, d.v.s.
\code{PI} är inte samma sak som \code{Pi}.



\section{Variabler}

En variabel är likt en konstant en identifierare, men istället för att
representera ett värde representerar variabeln ett minnesutrymme i vilket man
kan lagra data. De fungerar precis som konstanterna, med skillnaden att man
faktiskt kan ändra värdet som lagras i variabeln. Namnet uppfyller kravet för
en identifierare, men får inte vara enbart versaler (för det tolkar Python som
en konstant).
\begin{lstlisting}[style=text]
>>> x=5
>>> y=3
>>> z=x*y*3
>>> print x, y, z
5 3 45
>>> print x
5
>>> print y
3
>>> print z
45
>>> x=x+1
>>> print x
6
>>> print z
45
>>>
\end{lstlisting}
Notera att värdet på \code{z} inte ändras när vi ändrar värdet på \code{x},
detta för att det är värdet \code{45} som lagras i \code{z} och inte relationen
\code{x*y*3}.



\section{Typkonvertering}

Ibland kan man vilja konvertera vissa typer till andra. Det kan vara att man
läst in en sträng från tangentbordet och vill konvertera den till ett tag
(om användaren matade in ett tal). Ett exempel i Python-tolken:
\begin{lstlisting}[style=text]
>>> x="3.14"
>>> pi=float(x)
>>> print pi
3.14
>>> print x+2
Traceback (most recent call last):
  File "<stdin>", line 1, in ?
  TypeError: cannot concatenate 'str' and 'int' objects
>>> print pi+2
5.14
>>>
\end{lstlisting}

Ett annat bra exempel är procentberäkningar,
\begin{lstlisting}[style=text]
>>> 99/100*100
0
>>> float(99)/100*100
99.0
>>>
\end{lstlisting}
där man får fel svar om man inte explicit typkonverterar.



\section{Funktioner}

Funktioner utgör en viktig byggsten i programmeringen, liksom inom matematiken
kan den användas för att dela upp större problem i mindre och bidra till en
bättre ordning. De används dessutom på samma sätt.
\begin{align}
	\nonumber
	f(x) &= 2 \cdot g(x) + 3 \\
	\nonumber
	g(x) &= x^2
\end{align}
inom matematiken skulle kunna skrivas i Python som
\begin{lstlisting}
def g(x):
    return x*x

def f(x):
    return 2*g(x)+3
\end{lstlisting}

När vi talar om funktioner består de av flera delar, funktionen består av ett 
funktionshuvud och en funktionskropp. Funktionshuvudet är den första raden i
funktionsdefinitionen. Den består av funktionsnamn och formella parametrar.
D.v.s.
\[
\overbrace{\text{def functionname(}
	\underbrace{\text{argument1, argument2, ...}}_{\text{formella
		parametrar}}\text{):}}^{\text{funktionshuvud}}
\]
Resten av funktionen är funktionskroppen, d.v.s. all kod som hör funktionen
till.



\section{Ett programexempel}

Här kommer ett exempel på ett program, \verb'O1.py', som gör någonting vettigt.
Det beräknar kinetisk och potentiell energi för en boll.

\lstinputlisting[language=Python]{./O1.py}

En körning av detta program (från en terminal) kan se ut så här:
\begin{lstlisting}[style=text]
$ python O1.py
Ange bollens massa: 2
Ange bollens höjd: 3
Ange bollens hastighet: 4
Bollens rörelseenergi =  16.0 Joule
Bollens potentiella energi 58.92 Joule
$

\end{lstlisting}



\section{Tips och kommentarer}

Några punkter att tänka på:
\begin{itemize}
	\item Blanda aldrig språk, antingen skriver ni alla variabelnamn och
	funktioner på engelska eller på svenska -- aldrig båda!
	\item Använd förklarande namn till alla variabler och funktioner!
	\item Skriv kommentarer!
	\item Dela upp programmen i mindre delar -- funktioner! Det blir
	lättare att följa och det blir snyggare kod, men framför allt mycket
	enklare att programmera!
\end{itemize}

